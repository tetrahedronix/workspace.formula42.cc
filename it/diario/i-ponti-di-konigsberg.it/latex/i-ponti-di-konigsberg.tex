\documentclass[12pt,b5paper]{article}
\usepackage[utf8]{inputenc}
\usepackage[italian]{babel}
\usepackage{makeidx}
\usepackage{graphicx}
\usepackage{lmodern}
\usepackage{kpfonts}
\usepackage{hyperref}
%\author{Giulio Carlo}
\date{}
\title{I ponti di Königsberg}
\begin{document}
\maketitle
\begin{flushright}
\textit{A Ignazio Silone}
\end{flushright}
\textit{Incontro con un aspirante `connessionista' e con il suo famoso connazionale,
il professore Rigorenko Doppia N, giunto dalla Russia in Italia
alla ricerca del Santo Graal della conoscenza: un segreto nascosto dietro il
potere delle reti neurali artificiali, che potrebbe rivoluzionare il mondo
della tecnologia e della connettività globale}

\vspace{10mm}

Ricordo, come se fosse ieri, l'incontro che ho avuto con il Professore Rigorenko
Doppia N in occasione di un interessante seminario dal titolo ``Intelligenza
Artificiale Generativa: Rischi e Opportunità", organizzato dalla dottoressa
Eureka Di Grecia presso un'importante istituzione milanese e patrocinato da
un'azienda ai vertici nella produzione di microprocessori. Eureka è nota per la
sua passione nell'organizzare eventi che riuniscono i migliori esperti nel campo
dell'intelligenza artificiale e della linguistica computazionale.
Ripenso a quell'evento come a un'opportunità unica e imperdibile per espandere le
proprie conoscenze.

Avendo già collaborato con Eureka in passato, feci estendere l'invito al seminario a l'Operaio in
Blu, un novizio \textit{connessionista}, che avevo conosciuto in
gioventù quando entrambi frequentavamo il MIND. In quel periodo, l'Operaio si
occupava del montaggio delle attrezzature di criogenia per conto di una grande
azienda privata attiva nella ricerca sulla fisica delle particelle.  

Mentre preparavo i bagagli per il viaggio a Milano, ricevetti un
messaggio di posta elettronica dall'Operaio in Blu che mi annunciava la
presenza in città del suo connazionale, il Professore Rigorenko Doppia N,
un'autorità nel campo della linguistica computazionale. Il Professore è famoso
per le sue teorie audaci sull'intersezione tra linguistica, coscienza e
intelligenza artificiale e incontrarlo avrebbe significato confrontarmi con una
mente capace di spingersi oltre i confini convenzionali del sapere.

Il seminario non deluse le aspettative degli invitati e rivelò nuove idee e
discussioni interessanti, con presentazioni che spaziavano dalle prospettive
più teoriche alle applicazioni più pratiche dell'informatica, dando a tutti
molto materiale su cui riflettere. Dopo la sessione, l'Operaio suggerì di
prolungare il nostro soggiorno a Milano per alcuni giorni  in compagnia
del professore, per approfondire ulteriormente le nostre conversazioni sulle
applicazioni dei linguaggi di programmazione, la linguistica e l'intelligenza
artificiale.

Così, ci ritrovammo seduti in un accogliente albergo milanese di una calda 
estate, davanti a una finestra che si affacciava su Corso Vittorio Emanuele;
mentre la vita frenetica della città si svolgeva all'esterno, noi eravamo
immersi in una discussione appassionata sul futuro della tecnologia. 

Il nostro primo incontro si trasformò rapidamente in un vivace scambio di idee,
mentre rielaboravamo e ampliavamo le tesi presentate durante il convegno della
mattina. Fu una conversazione interessante: ci addentrammo negli intricati
meccanismi degli algoritmi, dove uno dei miei amici dimostrò una comprensione
sorprendentemente approfondita e un'esperienza unica. In modo quasi naturale,
il dialogo si spinse fino ai confini della coscienza artificiale, un
territorio che ci affascinava e ci sfidava allo stesso tempo. 

Ho ancora vivide le immagini e le parole di quel pomeriggio afoso, impresse
nella mia memoria come il resoconto di un giornalista che annota ogni dettaglio
con cura ad un evento straordinario a cui ha assistito.

Rivedo chiaramente il momento in cui entrai nella grande sala dell'albergo, 
c'era un'oasi di tranquillità interrotta solo dal leggero sibilo dell'aria
condizionata, e una tenue luce filtrava dalle tende a righe color crema.
Al centro, un tavolo di legno scuro dominava lo spazio, circondato da comode
poltrone. Le pareti erano adornate da opere d'arte astratte, creando un'atmosfera
contemporanea. Il tappeto di un verde cacciatore conferiva al salone un tocco di
eleganza, mentre un grande specchio appeso sopra il tavolo amplificava lo spazio,
dando l'impressione di un locale più ampio di quanto fosse in realtà. Un minibar
collocato discretamente in un angolo offriva una selezione di bevande e snack
per gli ospiti. Sul tavolo, accanto a un porta documenti, giaceva una
tavoletta elettronica di ultima generazione. Schermo luminoso e tasti sensibili
al tatto riflettevano la luce dell'ambiente circostante. 

Due figure si trovavano sedute di fronte al tavolo, come votate a un genio
tecnologico, silenzioso e predisposto a esaudire i loro desideri. Il Professor Rigorenko
Doppia N, con il suo sguardo penetrante era intento in una concitata conversazione
telefonica, mentre l'Operaio in Blu, con un'espressione di curiosità, osservava
attentamente la tavoletta.

L'Operaio era un uomo dall'aspetto curato e distinto. Nei contesti formali,
sfoggiava un gusto impeccabile, prediligendo abiti su misura che ne esaltavano la
corporatura. La sua passione per il colore blu si manifestava anche in questa
occasione: aveva scelto un completo gessato senza cravatta, accompagnato da un
gilet di lana azzurro. Le sue scarpe, attentamente selezionate e abbinate con il vestito, aggiungevano un tocco di classe al suo outfit. Gli occhiali, dal gusto
vintage con montatura dorata, presentavano lenti leggermente scure che attribuivano
un'aria di mistero al suo sguardo. Nonostante una capigliatura rada, manteneva
un aspetto raffinato. La barba, curata con precisione, suggeriva una certa
maturità, ma il suo fascino di persona vitale e gentile era rimasto intatto.
Di corporatura snella e di statura moderata, la sua carnagione chiara, tipica dei
caucasici dell'Est europeo, contrastava in modo affascinante il colore scuro dei
suoi occhi, una caratteristica rara tra i moscoviti.

In contrasto con l'Operaio, il professore era una figura imponente, molto alta
e slanciata. I suoi capelli bianchi, folti e ben pettinati, si incorniciavano
con un'elegante barba Van Dyke per conferire al suo volto un'aria distinta.
Indossava al polso un elegante orologio analogico argentato.
Il suo abbigliamento rifletteva un gusto personale per il classico, ma senza
scadere nel vistoso: un completo a tinta unita nero, scarpe di cuoio di un 
lucente colore corvino, una camicia grigia e una cravatta dai toni sobri. 
Gli occhiali con montatura nera accentuavano la sua fronte alta e trasmettevano
un'intelligenza straordinaria. I suoi occhi smeraldo viravano al grigio e
mettevano quasi soggezione. In generale, la sua presenza rievocava il
personaggio di Padre Lankester in ``L'Esorcista", con la stessa autorità
raffinata e la stessa profonda saggezza.

``Sì... Avevo ordinato un armadio con 8 rack da 22... Come dice, prego!?"
Prima che l'Operaio ci presentasse, il professore stava concludendo la 
trattativa telefonica. ``No, no.. non ha importanza il processore,
per me conta la memoria! La memoria! Si ricordi!"

``Professore, vi presento il Connessionista..."

``Scusate ma ho avuto una giornata impegnativa, nonostante i chilometri che mi
separano dal mio ufficio, al dipartimento di Leningrado"

``San Pietroburgo. Professore, S.A.N..P.I.E.T.R.O.B.U.R.G.O."

``...siamo tutti eccitatissimi perché proveremo un nuovo algoritmo per la 
generazione di immagini artistiche -- Connessionista hai detto? -- verranno a
studiarlo alcuni colleghi d'oltre cortina -- Pensavo che il \textit{connessionismo}
fosse il solito mito moderno che si diffonde a velocità virale su Internet -- e
l'oggetto sul tavolo..."

``No, siamo concreti come i pennelli di Dalì sulla sua tela più famosa,
\textit{La persistenza della memoria}, ma a differenza di un mito, vogliamo
dipingere il futuro e sfidare persino la fantasia più sfrenata.", risposi 
cercando di non apparire polemico.

``Ti presento il Professore Rigorenko Doppia N."

``Piacere di conoscerla Professore! Il suo lavoro precede la sua fama, è un onore
poterla finalmente incontrare di persona. Ho trovato particolarmente affascinante
la ricerca che ha presentato questa mattina al seminario. L'idea di applicare i
modelli linguistici all'analisi delle immagini mi ha colpito per l'eleganza della
sua soluzione e le implicazioni che potrebbe avere nel campo dello studio delle
opere d'arte"

``Il piacere è tutto mio, figliolo. Come accennavo al nostro amico comune qui 
presente, al momento mi trovo in viaggio per Parigi. Grazie a un accordo tra la
mia università e il prestigioso Museo del Louvre, avrò l'opportunità di
raccogliere dati fondamentali per l'algoritmo su cui stiamo concentrando le
nostre energie. Utilizzerò un strumento ottico dotato di una fotocamera speciale
per questo scopo, o meglio, come tutti al dipartimento di fisica amiamo definirlo,
la produzione di opere d'arte più originali delle originali."

``Ma a cosa serve?" chiese l'Operaio, il tono della sua voce vibrava di
interesse e il suo sguardo mostrava curiosità per l'oggetto sul tavolo.

Il Professor Rigorenko Doppia N guardò anche lui il portatile e sorrise
leggermente, con gli occhi socchiusi come se stesse meditando. Aveva sentore che
quella domanda prima o poi sarebbe arrivata, ma sapeva anche che era troppo
presto per fornire chiarimenti. 

``Questa è una domanda importante, compagno," rispose con calma, i suoi occhi
brillavano di un'intelligenza sottile. ``Ma prima, dobbiamo chiederci: a cosa
serviamo noi? Come comunichiamo?"

L'Operaio annuì, riflettendo sul commento del Professor Rigorenko, e sorrise
anche lui. ``Non avevo mai considerato questa prospettiva," ammise con genuino
interesse.

``Ma come potremmo vivere senza linguaggi?" chiese di nuovo, con un'espressione
di perplessità. ``È possibile?"

Evidentemente, la telefonata e la mia presentazione avevano interrotto la
loro discussione riguardo i linguaggi e il portatile sul tavolo sembrava essere
al centro di questa esplorazione teorica.

Il Professor Rigorenko Doppia N si prese un altro momento per meditare su
questa nuova domanda, mentre la luce del dispositivo illuminava il suo volto pensieroso. ``Farò un breve
\textit{briefing} per il nostro amico appena arrivato, riprendendo il mio discorso
da dove l'ho lasciato.", disse. "Ebbene, i linguaggi sono una parte fondamentale
della nostra esistenza. Non solo nei contesti di programmazione informatica come i
literate programming e WEB che stiamo per discutere, ma anche nella nostra comunicazione quotidiana. I linguaggi naturali si sono evoluti nel tempo, proprio
come la tecnologia. Questa evoluzione ha avuto un impatto significativo sulle
scoperte scientifiche: senza un linguaggio chiaro e preciso, molte delle nostre più
grandi innovazioni non sarebbero state possibili. La nostra comprensione del mondo e
della realtà è strettamente influenzata dalla capacità di comunicare attraverso
linguaggi, che continuano a svilupparsi in risposta alle nostre esigenze e alle
nuove tecnologie. Senza linguaggi, non potremmo trasmettere pensieri, emozioni o
conoscenze. La nostra comprensione del mondo e della realtà è strettamente
influenzata dalla capacità di comunicare attraverso linguaggi."

L'Operaio ascoltava attentamente il Professore, annuendo di tanto in tanto.
Dopo un momento di silenzio, disse: "Quindi, in un certo senso, i linguaggi
sono il fondamento della nostra esistenza." Si capiva che era sempre più
coinvolto nell'argomento.

Fissò per un istante il congegno del Professore sul tavolo, poi aggiunse: ``Se ci
pensiamo, ogni cosa che facciamo, ogni pensiero che formuliamo, passa attraverso
un linguaggio, che sia verbale, matematico o simbolico. È come se la realtà
stessa fosse codificata in una sintassi ancora da decifrare completamente."

Il Professore annuì, accennando un gesto con la mano, come se volesse rispondere
sull'argomento, ma si trattenne. Forse avrebbe approfondito il tema in un altro momento. 
Poi disse: ``Ma questo aspetto filosofico dei linguaggi non è l'argomento del nostro banchetto
di scienza odierno. Oggi rifletteremo per lo più sull'innovativo utilizzo della
literate programming, argomento che la dottoressa Eureka ha trattato al
convegno con il suo studio pionieristico sulla teoria dell'informazione come
strumento efficace nella ricerca scientifica e nello sviluppo tecnologico,"
tagliò corto il Professor Rigorenko Doppia N, mentre scrutava attentamente il
piccolo computer. Poi aggiunse: ``Attraverso la literate programming, possiamo integrare
il codice sorgente con la documentazione, rendendo il processo di sviluppo del
software più accessibile, rapido e comprensibile. Questa mattina, Eureka ha
mostrato un'applicazione sperimentale della literate programming davvero degna
di nota. Non mi sorprende sia stata proprio lei a collegare le nuove tecnologie
di riconoscimento vocale al processo di sviluppo del software in modo così
originale, dato che ho avuto la possibilità di studiare le sue ricerche. A
proposito, lei, Connessionista, che percorso ha intrappreso per arrivare fin qui?" 

Rigorenko raccolse il dispositivo dal tavolo e iniziò a digitare qualcosa con aria
distratta, ma era evidente che si aspettava una risposta. Non mi sentii affatto
offeso dalla sua domanda, anzi, la trovai comprensibile. Non ero certo un nome
noto nell'ambiente accademico, e la mia presenza lì poteva sembrare fuori
posto. Tuttavia, ero certo che l'Operaio gli avesse fornito una breve bio su di
me per indurlo a partecipare a questo incontro post-seminario, il che mi faceva
pensare che ci fosse un motivo più profondo dietro la sua curiosità. 
D'altronde quell'incontro non era certo un colloquio di lavoro, o forse in 
qualche modo lo era? 

Cercai di formulare il mio curriculum: sono stato ricercatore al MIND dal 1999 al 2005, e successivamente assistente
nel laboratorio LALALAB, sotto la direzione della professoresssa Eureka. Mi
occupavo principalmente di sperimentazione su nuovi linguaggi di
programmazione, con particolare interesse per le reti di Petri\footnote{Una rete di Petri è un modello matematico utilizzato per descrivere e analizzare il comportamento di sistemi distribuiti, particolarmente utile per rappresentare processi con flusso di dati o di risorse. Inventato da Carl Adam Petri negli anni '60, questo modello è spesso impiegato per lo studio di sistemi complessi e paralleli, come quelli informatici e industriali.} riflessive. Oggi
offro la mia esperienza per consulenze a centri di ricerca, istituzioni e
università, per qualsiasi cosa sui generis, spesso fuori dagli schemi. Inoltre, ho fondato e dirigo
la Scuola di Connessionismo... No, non risposi così al professore, invece preferii
esporre il mio passato di studioso in maniera più naturale e spontanea. Evitai
di snocciolargli un elenco cronologico di attività. ``Ho lavorato come ricercatore per
sei anni, poi sono stato assistente della professoressa Eureka, che lei già
conosce. Mi sono occupato di nuovi linguaggi di programmazione e ricerca sulle
reti di Petri riflessive. Oggi metto la mia esperienza al servizio 
di centri di ricerca, dando una mano su progetti un po' particolari.
Immaginavo che l'Operaio le avesse già accennato qualcosa al riguardo...",
risposi con un pizzico di innocente provocazione.

``Eccellente! Lei è proprio la persona più indicata per un progetto che a Leningrado
vogliamo avviare sull'intelligenza artificiale. Sapete, per mantenerci al passo con
i nostri cari antogonisti del blocco capitalistico anglo-americano, il governo sta
destinando fondi illimitati alla ricerca in questo settore e il nostro `aggregatore
di menti`, una base strategica di sessanta metri quadrati nei sotterranei della
Statale di Mosca, ha ricevuto la fetta più grande. Se lei è disponibile, la vedrò
con piacere a Mosca e, chissà, forse un giorno avrà l'opportunità di aprire la sua
prima scuola di connessionismo nel blocco orientale.", disse con un leggero sorriso
in segno di soddisfazione.

L'Operaio in Blu ascoltava attentamente i nostri discorsi, ma una domanda gli
balenò in mente. ``Professor Rigorenko," cominciò, l'espressione seria mentre
cercava di formulare le sue parole con precisione, ``mi chiedo, WEB non è molto
diffuso, come mai?"

Il Professor Rigorenko Doppia N annuì. ``È vero, WEB non ha ottenuto il successo
che meriterebbe," rispose. ``In parte perché richiede troppo tempo e impegno
per essere padroneggiato. Non si tratta solo di programmare, ma di scrivere
il codice come se fosse un racconto, un testo che altri possano leggere e
comprendere. E poi, molti programmatori preferiscono restare nei metodi che
conoscono, senza esplorare nuovi strumenti, anche se potrebbero migliorare il
loro lavoro."

Anche l'Operaio fece un cenno di assenso, convinto. ``Capisco," disse, ``Forse è
solo questione di tempo, prima che ci si renda conto delle potenzialità. Dopo
tutto, come lei ha sottolineato, il linguaggio è la chiave che unisce scienza e
tecnologia.".

Il Professor Rigorenko sorrise. ``Esattamente! Lasciatemi spiegare brevemente
il concetto di literate programming." Si spostò leggermente sulla sedia cercando
una posizione più comoda. ``È un'idea che Donald Knuth ha introdotto\footnote{Knuth,
Donald. Literate Programming, Center for the Study of Language and Information,
1992.} molti anni fa. Invece di scrivere solo codice, si scrive una storia: il
programma, i pensieri dietro le scelte, le spiegazioni di come tutto funziona.
È come se ogni riga di codice fosse accompagnata da una nota a margine che dice
'ecco perché ho fatto questo'."

Ripose il palmare sul tavolo, ma continuò a fissarne lo schermo, esaminando
le informazioni che vi scorrevano. Per un istante, nella stanza calò un silenzio
che né io né l'Operaio osammo interrompere. In quel momento, il Professore mostrò di poter 
riflettere su più aspetti contemporaneamente, come se stesse considerando non
solo le conseguenze di ciò che aveva appena detto o fatto, ma anche le implicazioni 
future delle sue parole e dei suoi gesti. Ed ecco che sollevò lo sguardo e riprese da dove aveva
lasciato il discorso: ``Knuth ha creato uno strumento per rendere possibile tutto questo, e negli anni il concetto è stato ripreso da
altri ricercatori. È qui che entra in gioco CWEB\footnote{Donald E. Knuth e Silvio
Levy, The CWEB System of Structured Documentation, 1994. Questo testo presenta
il sistema CWEB e discute il concetto di literate programming, evidenziando come
combinare codice sorgente e documentazione per una maggiore chiarezza e
comprensione}, una versione migliorata di WEB che permette di fare lo stesso con linguaggi
di programmazione più moderni. Il punto non è solo programmare, ma farlo in modo
che chiunque, anche una macchina, un'intelligenza artificiale, in futuro, possa
capire il tuo lavoro come se stesse leggendo un libro." 
%- Stop revisione

Il Professor Rigorenko aveva appena terminato di spiegare l'origine della
literate programming e del sistema CWEB quando un ricordo affiorò dal suo
passato. ``Parlando di Donald Knuth," cominciò, con una vena di nostalgia nel
tono della voce, ``mi viene in mente un incontro che abbiamo avuto molti anni fa,
durante un simposio internazionale sugli algoritmi organizzato da un caro amico
russo, Andrei Ershov. Eravamo in Unione Sovietica, in un'epoca in cui la scienza
era un raro terreno neutro di incontro tra le ideologie opposte."

L'Operaio sollevò lo sguardo con interesse, incuriosito dall'anticipo della
storia che il Professor Rigorenko stava per raccontare.

``Ershov era un amico di John McCarthy\footnote{John McCarthy è considerato il padre dell'intelligenza artificiale e l'inventore del linguaggio di programmazione Lisp, fondamentale per lo sviluppo dei primi sistemi di IA.} e frequentava spesso le feste organizzate
a casa di McCarthy in America," proseguì Rigorenko, sempre più immerso nei ricordi.
``Fu durante una di queste occasioni che Knuth espose un'idea che gli era
balenata in mente, una sorta di pellegrinaggio per celebrare l'origine degli
algoritmi nel luogo di nascita di Al-Khwārizmī, l'odierna città di Khiva che
si trova in Uzbekistan."

Il professore aggiunse: ``Mi trovavo in gita a Leningrado e fui contattato da
Ershov che mi avvisava dell'arrivo di Knuth. Donald era già molto popolare in
tutta l'Unione Sovietica, i suoi scritti circolavano in ambito accademico, seppure
non tutti avevano accesso alla documentazione scientifica occidentale... dovete
sapere", a questo punto la sua voce divenne quasi un sussurro,  ``c'erano
delle gerarchie da rispettare... molte domande a cui rispondere e molta carta
bollata da compilare."

``Io ero ricercatore a Mosca e mi interessavo di teoria degli algoritmi
approssimati, potete quindi immaginare che poter incontrare Knuth in persona
era un'occasione irripetibile che non potevo assolutamente perdermi.
Raccolsi i fondi necessari per il lungo viaggio in Uzbekistan, metà dei quali
provenienti dai parenti e il resto anticipato dall'Università Statale di Mosca
sul mio stipendio. La città di Khiva purtroppo ricadeva in una zona di massima
sicurezza soggetta a restrizioni, anche a noi russi era chiesto un permesso
speciale d'ingresso. Ma senza perdermi d'animo, mi ricordai che avevo risolto
un problema piuttosto difficile sui grossi elaboratori del KGB per conto di
un importante esponente del partito. Lo contattai una sera e il giorno seguente
ero già su un volo militare diretto all'Aeroporto di Urgench, con in tasca un
biglietto d'autobus per Khiva."

``Con l'animo più felice di tutta la Madre Russia," continuò il professore,
``presi un soggiorno in un albergo di Khiva. Non dovetti aspettare molto; il
giorno successivo mi trovai di fronte a Donald Knuth. Dopo essermi congratulato
con lui per la recente premiazione al Turing, il Nobel per i matematici,
instaurammo subito un proficuo scambio di vedute che segnò, per quanto mi
riguarda, una svolta nella mia carriera di informatico. Finora mi ero concentrato
principalmente sulla teoria degli algoritmi approssimati, un campo relativamente
poco esplorato ma di grande potenziale. Tuttavia, il contributo di Knuth alla
mia formazione non si limitò a una singola conversazione sugli algoritmi. La
sua introduzione alla literate programming aprì un nuovo mondo che mi era finora
sconosciuto, quello della programmazione strutturata, e ha letteralmente
rivoluzionato  il mio approccio allo sviluppo del software. Abituato a navigare
tra il disordine dello \textit{spaghetti code}, ho presto compreso i benefici di una
progammazione più sistematica e documentata. La literate programming e WEB non
solo hanno reso il mio codice più leggibile e comprensibile, ma hanno anche
aperto nuove prospettive sulla progettazione del software, consentendomi di
fare passi da gigante verso le migliori pratiche di sviluppo che poi ho
trasmesso ai colleghi e ai miei studenti a Mosca."

L'Operaio stava ascoltando attentamente, riflettendo sulle spiegazioni del
Professore Doppia N, ma senza volerlo lo interruppe: ``Mi sorge il dubbio...
esistono sistemi utilizzati da grandi aziende per affrontare progetti su vasta
scala, come quelli a cui accennava? Sembra un'impresa riuscire a gestire tutto
in modo così strutturato, con centinaia di programmatori che si accavallano nei
commenti e nelle linee di codice..."

L'Operaio non aveva tutti i torti. Avevo già in mente una spiegazione, ma prima
che potessi esporla, il Professore mi anticipò, come se avesse percepito i miei
pensieri: ``È una domanda pertinente. Naturalmente esistono sistemi
di gestione del codice ben collaudati che garantiscono che il flusso di lavoro
sia ordinato, evitando conflitti. Le grandi aziende utilizzano questi strumenti
per gestire progetti complessi, ma ciò che le distingue davvero è la loro
capacità di adattarsi alle sfide impreviste." Fece un pausa. Ormai
consideravo le pause del professore segnali di digressioni filosoffeggianti di
profonda natura. Non mi sbagliavo nemmeno questa volta. Il Professore continuò:
``Questa capacità di adattamento è presente sia nelle economie capitaliste che
in quelle pianificate, ma in modi diversi. Nel capitalismo, è la concorrenza a
spingere l'innovazione: ogni impresa è costretta a trovare soluzioni veloci e a
innovare per sopravvivere. Tuttavia," aggiunse, con una luce nostalgica negli 
occhi, ``non dobbiamo dimenticare che anche l'economia pianificata ha avuto i
suoi punti di forza. L'Unione Sovietica, per esempio, pur con tutti i suoi
difetti, è riuscita a compiere straordinarie imprese collettive, come il
programma spaziale, grazie alla capacità di centralizzare risorse e conoscenze.
In situazioni di crisi, come durante la Guerra Fredda, questo approccio ha
permesso di mobilitare l'intera nazione verso obiettivi comuni, senza disperdere
energie nella competizione interna. A volte, nei momenti critici, come ho potuto
essere  testimone diretto, non si tratta di avere la tecnologia più avanzata,
ma di saperla usare in modo ingegnoso." 
 
Fece un'altra pausa, osservandoci attentamente mentre sorseggiava il tè ormai
raffreddato, ``Questo modo di affrontare i problemi, mi ricorda un vecchio
enigma che mi ha affascinato fin da ragazzo e che ha tenuto impegnati
matematici e logici per secoli. Un problema che, in apparenza sembra semplice,
ma che nasconde una complessità straordinaria: il problema dei sette ponti di
Kaliningrad. È un classico esempio di problema... irrisolvibile."

Intervenne l'operaio: ``Professore, non è proprio corretto chiamarla Kaliningrad.
Era Konigsberg, prima della guerra. Anche se capisco il vostro richiamo 
nostalgico...".  Si riferiva al nome, attualmente vigente, assegnato alla città di
Königsberg nel 1946 in onore di Mikhail Kalinin, un politico sovietico.

Quindi, il professore fece un cenno di assenso e rassegnazione, riflettendo sul cambiamento dei tempi e sul significato dei luoghi che una volta conosceva così bene. Cercai di allontanare i sentimenti malinconici che trasparivano dal
suo sguardo con una domanda razionale: ``Non propriamente irrisolvibile,
professore. Eulero ha infatti trovato una soluzione al problema, dimostrando che
non esiste un percorso che attraversi tutti i ponti una sola volta. È una
soluzione, anche se negativa, alla domanda posta.".

``Ah, sì, esattamente!", sorridendo aggiunse: ``Potremmo  dire che è 'risolvibile
in quanto irrisolvibile'. La bellezza della soluzione di Eulero sta proprio
nella sua semplicità e nel suo rigore. Ma cari amici, vedete, il problema non
finisce lì. Eulero non si è fermato a dire `non si può fare'. La sua analisi ci
ha regalato qualcosa di molto più prezioso: la nascita della \textbf{teoria dei grafi}."

Il Professore prese un tovagliolo dal tavolo e iniziò a disegnare delle linee
con gesti decisi.

``Quindi immaginate di voler attraversare tutti i ponti di una città una volta
sola, senza mai ripassare su uno di essi. Un esercizio che sembra lineare... ma
vi renderete presto conto che non esiste un percorso possibile. Ecco cosa
intendo per strutturare qualcosa di complesso. Anche la soluzione apparentemente
più semplice può nascondere una trappola logica."

L'Operaio in Blu si chinò in avanti, affascinato. Il professore continuò:
``Voglio portarvi oltre il problema dei ponti. Pensate a come possiamo estendere
questo tipo di problemi in contesti più complessi, come nei \textbf{cammini euleriani} o nei \textbf{circuiti hamiltoniani}\footnote{Un circuito hamiltoniano è un percorso in un grafo che visita ogni vertice esattamente una volta e ritorna al vertice di partenza. Deve il suo nome a William Rowan Hamilton, che studiò il problema nel XIX secolo}. Oppure, perché no, applicare queste
idee al \textbf{problema del commesso viaggiatore}\footnote{Il problema del commesso viaggiatore è un problema di ottimizzazione in cui, dato un insieme di città e le distanze tra di esse, si cerca il cammino più breve che visita ogni città esattamente una volta e ritorna al punto di partenza. È noto per la sua complessità computazionale.}, che è un caso ben più complicato,
dove non cerchiamo più solo un cammino che attraversi ogni ponte una volta, ma
il cammino più breve per visitare una serie di città e tornare al punto di
partenza."

``Interessante, quindi vedete un legame tra i due problemi?", chiesi. La
capacità del professore di trovare collegamenti improbabili tra teorie e
problemi mi impressionava, ma cercai di non sembrare troppo inquisitorio con le
mie domande.

``Esattamente! Se vogliamo trovare soluzioni ottimali ai problemi complessi,
dobbiamo partire da problemi semplici e costruire su di essi. Eulero ci ha
insegnato a guardare i grafi non solo come schemi di collegamenti, ma come
modelli per la complessità della realtà stessa. E non è forse ciò che fa il
il vostro 'club', il connessionismo? Creare connessioni, tessere reti di
significato?"

%- Revisione: aggiungere una digressione con un dialogo Connessionista-Professore
%- Il Connessionista potrebbe esporre tutto ciò che c'è da sapere sul 
%- Connessionismo (vd. appunti a "Superintelligenza", Bostrom)

Operaio in Blu curioso, chiese: ``Ma professore, tornando ai ponti... c'è davvero
una soluzione possibile?"

Rigorenko sorridendo rispose: ``Caro compagno, certo! Basta... aggiungere un
ponte! O, se preferisci, eliminare uno dei nodi con un grado dispari. Ma
ricorda: non si tratta solo di trovare una soluzione meccanica, ma di capire
il perché. Solo comprendendo la struttura del problema possiamo davvero
padroneggiarlo."

Dall'entusiasmo che iniziava a trasparire dal tono della sua voce, ero sicuro
che il professore stesse per raccontarci qualche suo aneddoto interessante.

``Il problema dei ponti di Kalin... Königsberg può apparire in contesti
impensabili. Spesso si manifesta in situazioni dove nessuno si aspetterebbe che
possa essere rilevante. Prendete, ad esempio, le complesse connessioni di un ecosistema marino. Chi potrebbe immaginare che lì, tra correnti d'acqua e creature
abissali, si nasconda un problema di cammini e ponti?". Aggiunse: ``Ogni ponte
è come un legame tra due, ma in realtà... non possono mai veramente connettersi
completamente. È impossibile attraversarli tutti una volta sola senza tornare
sui propri passi." 

``Un po' come le tensioni irrisolvibili tra potenze globali..." suggerii, intuendo
probabilmente dove volesse portare il discorso.

``Esattamente! Come un'equazione diplomatica, dove ogni parte in gioco ha i suoi
interessi e segreti, e nessuna via di uscita sembra davvero praticabile. Ma
sapete, a volte, nel caos apparente delle connessioni... possono nascere momenti
inattesi di collaborazione. Anche tra nemici."

Il Professore si fermò, il suo sguardo si oscurò per un attimo, mentre smise
di disegnare. Il tovagliolo scarabocchiato sembrò riportarlo a un ricordo
lontano, e con tono più grave disse: ``Parliamo sempre di tensioni tra Est e
Ovest, di conflitti irrisolti. Ma vi racconterò una storia... qualcosa che è
rimasto nascosto, fino ai tempi più recenti. Era il 1986. Il sottomarino K-219,
un progetto segreto della nostra Marina, affondò nell'Atlantico settentrionale.
Un disastro che avrebbe potuto innescare un conflitto globale. Ma quello che
nessuno sa... è che dietro le quinte, accadde qualcosa di straordinario."

Fece una pausa, guardandoci come per valutare se fosse il caso di
proseguire e rivelare un segreto di stato. Dopodiché, ci raccontò davvero qualcosa di incredibile:

``L'incidente del K-219 non fu solo una tragedia militare, ma anche un momento di
collaborazione segreta tra due superpotenze in piena Guerra Fredda. Il KGB e la
CIA, apparentemente su fronti opposti, lavorarono insieme per evitare il
disastro completo. Ho partecipato personalmente a una delle ultime operazioni
congiunte... un'operazione che, se fosse venuta alla luce, avrebbe cambiato il
corso della storia. E tutto questo... accadde prima del crollo del muro di
Berlino. Quel momento fu come un fragile ponte tra due mondi in conflitto."

L'Operaio in Blu si chinò in avanti, intrigato dalla piega presa dalla storia.

Il professore continuò: ``Fui contattato una notte da un numero sconosciuto
proveniente dalla Repubblica Socialista di Georgia. Era Yuri Andropov, uno dei
portavoce più noti del PCUS, con il quale condividevo un'amicizia all'interno
del partito. Yuri mi informò che avrei presto ricevuto una richiesta di
consulenza da parte del Ministro degli Affari Esteri, il georgiano Eduard
Shevardnadze. Un'importante figura comunicativa durante il mandato di Gorbachev,
Shevardnadze era noto in tutta l'Unione Sovietica e all'estero per il suo carisma
e per le sue azioni diplomatiche spesso clamorose e teatrali. Non potevo certo
rifiutarmi; avevo una tessera di partito vidimata e sapevo che Yuri sceglieva i
suoi collaboratori tra i più disponibili nel Politburo. Non volevo mettere il
mio amico in una situazione di malinteso con Yuri, così accettai senza
esitazione il suo consiglio e iniziai a preparare le valigie per un viaggio
oltreoceano."

Ricordavo bene quell'incidente. Per giorni occupò le prime pagine dei giornali,
i rotocalchi e i programmi televisivi in tutto l'Occidente e, in particolare,
in Italia. Solo pochi mesi prima, eravamo stati colpiti dalle piogge radioattive
provenienti da Chernobyl. Così, dissi al professore: ``L'incidente del K-219,
insieme al disastro di Chernobyl, ha profondamente sensibilizzato gli italiani.
Non sorprende che questi eventi abbiano avuto un'influenza determinante sul
referendum abrogativo sull'energia nucleare, che si tenne qualche mese dopo.
Probabilmente, ne influenzarono l'esito in modo decisivo."

Il Professor Rigorenko attese pazientemente che terminassi il mio discorso,
mentre osservava con attenzione come cercassimo di mettere insieme i pezzi
del puzzle. ``Volete sapere come tutto questo si collega?" chiese con un sorriso
appena accennato. ``I due incidenti, pur appartenendo a contesti diversi,
possono essere legati dal concetto di \textbf{decadimento delle connessioni},
sia a livello tecnologico che umano."

Cominciai a intuire la complessità dell'intreccio proposto dal professore: 
``La teoria dei grafi diventa lo strumento per analizzare e spiegare queste
connessioni spezzate, mostrando come ogni decisione e ogni errore influenzi
il resto del sistema.", osservai.

``Ha trovato il punto focale. Il sistema del K-219, e quello di Chernobyl prima
di esso, erano come reti con nodi (decisioni, tecnologie, persone) legati da
archi (protocolli, comandi, informazioni), complessi ma allo stesso tempo fragili.
Un solo guasto in un nodo compromette l'intero sistema. Proprio come nei ponti
di Königsberg, una volta attraversato un ponte in modo errato
(una decisione sbagliata), non si può tornare indietro senza compromettere
l'intera missione (attraversarli tutti una sola volta). Quindi, il problema dei
ponti di Königsberg non è soltanto una curiosità matematica. È un principio che
può essere applicato ovunque ci sia la necessità di gestire connessioni e flussi
di movimento. Ed è proprio qui che entra in gioco di nuovo la mia collaborazione
con un amico americano. Ma andiamo con ordine." Si portò la tazza del tè alla
bocca, bevendo l'ultimo sorso prima di proseguire.

``Partecipavo a una trasferta con altri scienziati provenienti da discipline
diverse, come scoprii meglio conoscendo i miei compagni lungo il viaggio.
C'erano fisici nucleari, chimici, matematici, ecologi e biologi. Il volo da
Mosca all'America non fu reso pubblico, né tantomeno festeggiato alla partenza
come una missione per onorare la nostra grande terra sovietica. Seguimmo invece
una rotta insolita, con scali in piccoli aeroporti, sperduti tra deserti
dell'Asia e la costa occidentale degli Stati Uniti. Ma ciò che mi colpì di più
fu la destinazione: un aeroporto militare dell'Arizona. Lì, senza troppe
formalità, ci chiesero di salire immediatamente su un aereo militare del
trecentocinquesimo aviotrasportati, senza fare domande. Infine, cosa ancora
più sorprendente, non ci fu consegnato alcun visto, né fummo controllati dagli
agenti dell'FBI."

Era calato un silenzio profondo nella hall dell'albergo, nonostante fossero solo
le sedici del pomeriggio. Il professor Rigorenko stava raccontandoci la sua storia
stupefacente, e mi sembrava che chiunque nella sala fosse altrettanto rapito dai
ricordi del 1986, a cui Rigorenko stava dando forma con le sue parole. Chiesi con
rispetto: ``Professore, state dicendo che questo viaggio misterioso, con una squadra
così eclettica di scienziati, non era stato annunciato né autorizzato ufficialmente?
Nemmeno in America? Ma qual era il vero scopo di tutta questa segretezza? Non riesco
a immaginare cosa potesse collegare biologi, fisici e matematici in una missione
tanto riservata."

La risposta del Professore non si fece attendere: ``Capisco il vostro sconcerto.
Anche io fui scioccato inizialmente nell'osservare la collaborazione così stretta tra americani e sovietici, come tra persone che si ritrovano ogni sera al
pub per un drink. Eppure, inspiegabilmente, a livello politico, nessuno sapeva nulla, tranne, naturalmente, Shevardnadze, il suo staff e qualche misterioso personaggio dell'élite americana, talmente potente da farci entrare senza visto."
%--
``Le faccio portare altro tè professore o preferisce un aperitivo?", chiesi con
gentilezza. 

``La ringrazio, ha notato che bevo molto tè. Ora ci vorrebbe una qualità leggera,
va benissimo verde con del bergamotto...", detto questo, riprese subito la sua
storia da dove l'aveva interrotta: ``Il viaggio verso la destinazione finale fu un
misto di adrenalina e inquietudine. La hostess americana in tuta militare ci
informò, senza giri di parole, che saremmo atterrati a Miami, ma non ci
sarebbe stata alcuna autorizzazione per lasciare l'aereo, e chiuse il discorso
aggiungendo che lo scalo successivo sarebbe stato l'aeroporto di St. David's,
nelle Bermuda. Così atterrammo a Miami solo il tempo necessario per rifornire
di carburante l'aereo. Durante il volo successivo, osservavo dal finestrino le
nuvole che si alternavano a squarci di terra e poi a squarci di mare, riflettendo sul
destino che mi attendeva. La mia mente correva tra le possibili implicazioni di
quella missione, che avevo ribattezzato, per intimo diletto, operazione
\textit{anglosovietica}: cosa avrebbero potuto chiedermi di fare? Quali segreti sarebbero emersi dai colloqui con le sfere militari straniere? Cosa non avrei
dovuto rivelare del mio lavoro per nessuna ragione al mondo?"

Questa volta fu l'Operaio in Blu a interrompere il professore: ``Non posso fare a
meno di chiedermi: come si sentiva in quella situazione così delicata? C'era paura?
La destinazione doveva suggerirvi che la missione c'entrasse con la tragedia del K-219, l'avevate già intuito?"

Il professore annui. ``Sì, c'era naturalmente paura e sì, affiorava qualche racconto a mezza voce sul K-219, che da qualche parte nell'Atlantico, come una bestia ferita, lottava contro il risucchio dell'abisso. Ma c'era anche consapevolezza che stavamo partecipando a
qualcosa di più grande di noi. L'idea di poter contribuire a una de-escalation tra
le superpotenze e la possibilità di essere i protagonisti era elettrizzante."

``E cosa vi dissero quando arrivaste?" chiese ancora il mio amico Operaio,
affascinato quanto me dalla piega assunta dalla storia.

``L'aeroporto era chiuso al traffico civile. Scesi dall'aereo fummo subito scortati
in una grande sala conferenze, dove ci aspettavano alcuni alti funzionari americani
con la loro squadra di scienziati che ci aveva preceduto.
Tra di loro notai un agente con il cartellino della CIA, un `pezzo grosso`, che fece una
chiara presentazione dell'accaduto e spiegò il nostro incarico segreto. Al
suo fianco riconobbi un commissario politico del mio paese che conoscevo di vista.
Sentivo l'aria carica di tensione. Eravamo sedici in tutto, tra americani e
russi. Per stemperare la situazione e facilitare la collaborazione, ci chiesero
di presentarci brevemente. Quando arrivò il mio turno, sentii un familiare
fruscio accanto a me. Guardai alla mia sinistra e incontrai lo sguardo di
John McCarthy, il visionario dell'intelligenza artificiale e creatore di Lisp. Ci eravamo già conosciuti anni prima, durante la commemorazione
della nascita di Al-Khwārizmī, ma rivederlo lì, in un contesto così diverso,
mi colpì. Il suo volto, solitamente sereno, era segnato dalla gravità della
situazione. Lui annuì, riconoscendomi e intuendo le mie preoccupazioni."

Il professore si fermò un attimo, sorseggiando il tè verde appena portato, prima di
continuare: ``Sapevamo entrambi che quella collaborazione tra KGB e CIA, seppur nascosta,
avrebbe potuto riscrivere la storia. E ogni secondo trascorso lì sembrava un passo
verso un abisso ignoto. La tensione era palpabile, ma c’era anche una strana
eccitazione. La storia si stava scrivendo sotto i nostri occhi.”

``Che è questo schiamazzo?", gridò improvvisamente qualcuno nella hall
dell'albergo, distraendoci e interrompendo il filo del discorso di Rigorenko. Mi
alzai, affacciandomi alla finestra vidi un gruppo di persone in coda di fronte
a un negozio lungo Corso Vittorio, in attesa della prima vendita di qualche
gadget alla moda. Chiusi la finestra e tirai le tende, isolandoci dal frastuono
esterno. Mentre lo facevo, riflettevo sul discorso del Professore: un'alleanza
segreta tra KGB e CIA... Il solo pensiero mi inquietava e affascinava allo stesso
tempo. Cosa significava davvero riscrivere la storia? Quante verità erano state
occultate dietro accordi impensabili? Mi voltai verso Rigorenko, cercando nel
suo sguardo un indizio su quanto ancora fosse disposto a rivelare, e mi
riaccomodai accanto ai miei amici. Quindi il professore continuò la sua storia con nuova lena: ``Durante
la crisi del K-219, ci trovammo di fronte a una situazione di emergenza
ambientale e militare. Il comparto missilistico danneggiato del sottomarino
minacciava il reattore nucleare, con il rischio di rilasciare materiale
radioattivo nell'oceano e mettere a rischio l'intero ecosistema marino.
Se il danno si fosse esteso al reattore, le correnti sottomarine avrebbero potuto
disperdere la contaminazione in ogni direzione, rendendo impossibile contenerla.
Fu in quel momento che io e John ci rendemmo conto che il problema dei ponti di
Königsberg poteva fornirci una soluzione."

Il professore afferrò un nuovo tovagliolo e riprese a disegnare. ``Immaginate che ogni
corrente marina sia come un ponte, e che le zone critiche dello scafo siano le terre
da attraversare. Se fossimo riusciti a tracciare un percorso che minimizzasse
l'interazione tra queste correnti, avremmo potuto confinare la contaminazione
radioattiva, impedendo che si diffondesse in modo irreparabile."

L'Operaio in Blu sobbalzò in avanti, sempre più sbalordito. ``Quindi, avete
usato una sorta di rete di Petri per contenere il danno?"

``Esatto!" esclamò il professore, ora con tono entusiasta. ``Grazie all'algoritmo
basato sul problema dei ponti di Königsberg, io e McCarthy sviluppammo una
simulazione in grado di prevedere come e dove si sarebbe diffusa la contaminazione
sul fondo marino. Utilizzando la teoria dei grafi, riuscimmo a isolare le aree
più critiche e a indirizzare le operazioni di contenimento verso i punti chiave,
impedendo un disastro ecologico di proporzioni enormi. Ma la cosa più sorprendente,
cari amici, è che questi calcoli ci furono chiesti a bordo del K-219 e li
eseguimmo con calcolatori antiquati, mentre il sommergibile si inabissava nelle
oscure e gelide acque dell'Oceano Atlantico!"

\newpage
\section*{Nota dell'autore}
Questo racconto è stato creato con l'assistenza dell'IA ChatGPT. Alcune parti
del testo sono state generate utilizzando la tecnologia di ChatGPT, mentre
altre sono state elaborate e integrate dall'autore. L'IA è stata utilizzata
come strumento creativo per arricchire e sviluppare il racconto in modo
originale. È un omaggio dell'autore a due geni della scienza dell'informazione:
Donald Knuth, padre dell'analisi degli algoritmi e creatore di TeX, e
John McCarthy, pioniere dell'intelligenza artificiale e ideatore del linguaggio
Lisp.

\selectlanguage{english}
\section*{GNU Free Documentation License}
    Copyright (C)  2024 Giulio Carlo.
   
    Permission is granted to copy, distribute and/or modify this document
    under the terms of the GNU Free Documentation License, Version 1.3
    or any later version published by the Free Software Foundation;
    with no Invariant Sections, no Front-Cover Texts, and no Back-Cover Texts.
    A copy of the license is included in the section entitled ``\href{https://www.gnu.org/licenses/fdl-1.3.en.html}{GNU Free Documentation License}"
    online.
\end{document}