\chapter{Il campo reale}
\section{Il prodotto cartesiano}
Siano $A,B$ sottoinsiemi di numeri reali, non vuoti.
\begin{defin}{}{}
Il \textbf{prodotto cartesiano} di $A$ e $B$ è l'insieme costituito da tutte le coppie ordinate $\lbrace(a,b),(b,a):a\in A, b\in B\rbrace$; questo insieme è indicato con $A\times B$:
\[
A\times B=\lbrace(a,b): a\in A, b\in B\rbrace.
\]
Qualora $A$ e $B$ coincidano si scrive $A\times A=A^2$.
\end{defin}
Si consideri il prodotto cartesiano ${\rm I\!R}\times {\rm I\!R}$ o ${\rm I\!R}^2$:
\[
{\rm I\!R}^2=\lbrace(a,b): a,b\in{\rm I\!R}\rbrace,
\]
su questo insieme valgono le operazioni di somma e prodotto tra coppie di numeri reali.
\begin{defin}{}{}
Date due coppie di numeri reali $(a,b), (c,d)$ si definisce \textbf{somma tra coppie} l'operazione $+$ che produce la sequenza $(a+c, b+d)$:
\[
(a,b)+(c,d)=(a+c,b+d).
\]
Si definisce \textbf{prodotto tra coppie} l'operazione $\cdot$ che produce la sequenza $(a\cdot c-b\cdot d,a\cdot d+b\cdot c)$:
\[
(a,b)\cdot(c,d)=(ac-bd,ad+bc).
\]
\end{defin}
\begin{remark}
Il simbolo $\cdot$ del prodotto tra coppie può essere omesso se non crea nessun equivoco. Per cui
\[
(a,b)\cdot(c,d)=(a\cdot c-b\cdot d,a\cdot d+b\cdot c)=(ac-bd,ad+bc).
\]
\end{remark}
\begin{remark}
L'insieme ${\rm I\!R}^2$ è un campo con due operazioni di somma e prodotto, quindi $({\rm I\!R}^2, +,\cdot)$, rispetto alle quali vale la proprietà commutativa e in ${\rm I\!R}^2$ ogni elemento diverso da zero possiede inverso .
\end{remark}